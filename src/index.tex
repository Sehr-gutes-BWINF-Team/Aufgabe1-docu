\maketitle
\tableofcontents

\vspace{0.5cm}


\section{Lösungsidee}\label{sec:losungsidee}
Das Satzfragment wird eingelesen und in eine Regular Expression überführt.
Außerdem wird das Buch von seinen Satz- und Sonderzeichen sowie Einrückungen bereinigt.
Anschließend wird mittels der Regular Expression alle in betracht kommenden Textstellen gesucht
und letztendlich Ausgegeben.


\section{Umsetzung}\label{sec:umsetzung}
Die Implementierung erfolgt in Python. \\
Es wird das Python re module verwendet.
Dieses ermöglicht es Reguläre Ausdrücke (Regular Expressions, Regex) in Python zu verwenden. \\
Zunächst wird das Buch eingelesen und bereinigt.
Dazu werden verschiedene Regular Expressions sowie Python String Manipulation verwendet.
Danach wird das jeweilige Textfragment eingelesen, und daraus ein entsprechender Regex geformt, welcher
auf alle Textstellen

\subsection{Buch bereinigen}\label{subsec:buch-bereinigen}
Im ersten Schritt wird alles, was kein Wort oder ein Leerzeichen ist entfernt. \\
Dies passiert mittels folgendem Regex: \verb/[^\w\s]/ \\
Die re\#sub Funktion findet und ersetzt alle Textstellen, welche vom angegeben Regex gefunden werden. \\
Anschließend werden alle Neuzeilen durch Leerzeichen ersetzt.
Zudem werden alle Einrückungen mittels Tabulator-Taste sowie Unterstriche entfernt.
Dafür wird Pythons String\#replace Methode verwendet.\\
Abschließend werden alle überflüssigen Leerzeichen mittels diesem Regex \verb/\s{2,}/ gefunden und durch ein einzelnes
Leerzeichen ersetzt.


\subsection{Textfragment finden}\label{subsec:textfragment-finden}



\section{Beispiele}\label{sec:beispiele}
Genügend Beispiele einbinden!
Die Beispiele von der BwInf-Webseite sollten hier diskutiert werden, aber auch eigene Beispiele sind sehr gut - besonders wenn sie Spezialfälle abdecken.
Aber bitte nicht 30 Seiten Programmausgabe hier einfügen!


\section{Quellcode}\label{sec:quellcode}
Unwichtige Teile des Programms sollen hier nicht abgedruckt werden.
Dieser Teil sollte nicht mehr als 2 bis 3 Seiten umfassen, maximal 10.
